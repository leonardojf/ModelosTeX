%       @file: AvaliacaoDeApresentacao.tex
%     @author: Guilherme N. Ramos (gnramos@unb.br)
%
% Modelo para avaliação de apresentação de artigos.
%
% Este documento usa a classe UnBExam, disponível em:
% https://github.com/gnramos/UnBExam

\documentclass{UnBExam}%

\documento{Avaliação Crítica}%
\professor{Luís P. F. Garcia}%
\renewcommand{\outrasInformacoes}{}%
\data{}%
\usepackage{multirow}%

\begin{document}%
	\begin{description}
		\item[Artigo:] \hrulefill\\\hrule\vspace{1em}%
		\item[\hspace{1.45em}Apresentação:] (\hspace{2.5em}) \hfill%
			\textbf{Conteúdo:} (\hspace{2.5em}) \hfill%
			\textbf{Organização:} (\hspace{2.5em})\\%
			\hspace*{-2.5em}\textbf{Recursos Visuais:} (\hspace{2.5em}) \hspace{4.1em}%
			\textbf{Aprendizado:} (\hspace{2.5em})
			\hfill \emph{Apenas valores de 0 - 10.}\vspace{1em}%
			\item[Comentários e Análise Crítica:] \hrulefill\\\hrule
	\end{description}

    \noindent
    \null\hrulefill\\\null\hrulefill\\\null\hrulefill\\
    \null\hrulefill\\\null\hrulefill\\\null\hrulefill\\
    \null\hrulefill\\\null\hrulefill\\\null\hrulefill\\
    \null\hrulefill\\\null\hrulefill\\\null\hrulefill\\
    \null\hrulefill\\\null\hrulefill\\\null\hrulefill\\
    \null\hrulefill\\\null\hrulefill\\\null\hrulefill\\
    \null\hrulefill\\\null\hrulefill\\\null\hrulefill\\
    \null\hrulefill\\\null\hrulefill\\\null\hrulefill\\
    \null\hrulefill\\\null\hrulefill\\\null\hrulefill\\
    \null\hrulefill\\\null\hrulefill\\\null\hrulefill\\
    \null\hrulefill\\\null\hrulefill\\\null\hrulefill
    \newpage%

    \begin{center}%
    	\textbf{TAMD - Avaliação de Apresentação}%
    \end{center}%

    Estas informações servem para avaliar a análise crítica da audiência e fornecer
    um retorno ao apresentador, de modo que este possa aprimorar suas habilidades.
    Nenhum aluno terá acesso às avaliações de outros, apenas a média resultante e
    comentários pertinentes (se houver) serão informados ao palestrante.%

    Para cada item a seguir, dê uma nota de 0 a 20 para cada item conforme sua
    percepção da apresentação. São 100 pontos possíveis, mas um total inferior a
    65 é considerado insuficiente.%

	\begin{description}%
		\item[Apresentação:] analisar se o volume de informações e a velocidade
		de apresentação foram adequados ao tempo designado; se a fala e entonação 
		foram variados para enfatizar pontos de interesse; se a postura, contato
		visual, expressões e gestos foram apropriados; se o linguajar foi
		objetivo e claro.
		\item[Conteúdo:] analisar se há foco claro e consistente, com transições
		apropriadas entre as ideias principais; se o apresentador demonstra
		conhecimento substancial do assunto; se as ideias principais são apoiadas
		por resultados científicos; se o palestrante comunica-se de forma efetiva
		com a audiência.
		\item[Organização:] analisar ser a sequência é apropriada (início, meio
		e fim); se a fala é clara, fluida e segue uma sequência lógica; se o
		conteúdo é apresentado com diferentes materiais de apoio; se a mensagem
		é apresentada claramente.
		\item[Recursos Visuais:] analisar se os recursos são bem feitos e se
		melhoram a apresentação; se há uso correto da tecnologia; se há conexão
		clara entre os recursos visuais e a mensagem; se é evidente que houve
		interesse e criatividade na criação.
		\item[Aprendizado:] analisar se o apresentador é capaz de responder de
		forma correta e inteligente às questões; se demonstra conhecimento do
		assunto; se explica adequadamente o desenvolvimento da mensagem; se
		é capaz de refletir criticamente sobre o conteúdo apresentado.
	\end{description}%
	Comentários [construtivos] são livres. Na análise crítica posicione-se em
	relação as seguintes questões \emph{sobre o conteúdo apresentado}:
	pertinência no contexto da disciplina; forma como foi abordado; comparação
	com outras abordagens do mesmo assunto (caso conheça).
\end{document}%
